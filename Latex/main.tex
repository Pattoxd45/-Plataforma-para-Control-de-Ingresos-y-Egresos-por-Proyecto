\documentclass[12pt, a4paper]{article}
\usepackage[spanish]{babel}
\usepackage[utf8]{inputenc}
\usepackage{graphicx}
\usepackage{hyperref}
\usepackage{enumitem}
\usepackage{parskip}
\usepackage{float}

\title{Informe Técnico: Plataforma Fintrax}
\author{Carlos Ulloa, Patricio Valdés, Víctor Sepúlveda}
\date{Junio 2025}

\begin{document}

\maketitle
\newpage
\section{Introducción}
Fintrax es una plataforma integral diseñada para el control de ingresos y egresos por proyecto, dirigida a emprendedores, freelancers, startups y organizaciones que requieren una gestión financiera precisa desde etapas tempranas. Este informe detalla las funcionalidades, metodología de desarrollo, arquitectura, y contribuciones individuales del equipo.
\newpage
% insertamos las funciones
\section{Funcionalidades}
\subsection{Gestión de Usuarios}
\begin{itemize}
    \item \textbf{Autenticación} mediante SupaBase Auth:
    \begin{itemize}
        \item Registro con verificación de email
        \item Inicio de sesión con email/contraseña
        \item Cierre de sesión con invalidación de tokens JWT
        \item Sesiones persistentes seguras
    \end{itemize}
    
    \item \textbf{Recuperación de contraseña}:
    \begin{itemize}
        \item Flujo automatizado con SupaBase
        \item Enlaces de recuperación con expiración
        \item Validación de seguridad en el frontend
    \end{itemize}
    
    \item \textbf{Perfil de usuario}:
    \begin{itemize}
        \item Almacenamiento en tabla \texttt{auth.users} de SupaBase
        \item Edición segura de información básica
    \end{itemize}
\end{itemize}

\subsection{Gestión de Ingresos y Egresos}
\begin{itemize}
    \item \textbf{Registro de transacciones}:
    \begin{itemize}
        \item Implementado mediante tablas \texttt{transacciones} en SupaBase
        \item Campos obligatorios: monto, fecha, descripción, categoría, tipo (ingreso/egreso)
        \item Relación con proyectos mediante \texttt{id\_proyecto}
        \item Validación en tiempo real con SupaBase RPC
    \end{itemize}
    
\item \textbf{Edición y eliminación}:
\begin{itemize}
    \item \textbf{Protección de datos}:
    \begin{itemize}
        \item Implementación de Row Level Security (RLS) para restringir acceso
        \item Políticas personalizadas para UPDATE/DELETE en SupaBase
    \end{itemize}
    
\item \textbf{Soft Delete}:
\begin{itemize}
    \item \textbf{Implementación básica}:
    \begin{itemize}
        \item Campo \texttt{deleted\_at} (TIMESTAMPTZ) en tabla \texttt{transacciones}
        \item Valor por defecto NULL (registro activo)
        \item Índice parcial para optimizar queries: \texttt{WHERE deleted\_at IS NULL}
    \end{itemize}
    
    \item \textbf{Eliminación por errores}:
    \begin{itemize}
        \item Columna adicional \texttt{deletion\_reason} (TEXT) con valores:
        \item \quad 'error' - Registro incorrecto
        \item \quad 'normal' - Eliminación estándar
        \item \quad 'duplicado' - Registro repetido
    \end{itemize}
\end{itemize}
    \item \textbf{Histórico de cambios}:
    \begin{itemize}
        \item Tabla \texttt{transacciones\_log} para auditoría
        \item Triggers en SupaBase que registran:
        \begin{itemize}
            \item Usuario que realizó el cambio
            \item Fecha y hora exacta
            \item Valores anteriores (para operaciones UPDATE/DELETE)
        \end{itemize}
    \end{itemize}
    
    \item \textbf{Validaciones}:
    \begin{itemize}
        \item Confirmación en UI para operaciones críticas
        \item Límite de tiempo para reversión de operaciones
    \end{itemize}
\end{itemize}
    
    \item \textbf{Filtrado avanzado}:
    \begin{itemize}
        \item Uso de SupaBase \texttt{select()} con filtros dinámicos
        \item Posibilidad de filtrar por:
        \begin{itemize}
            \item Rango de fechas
            \item Monto mínimo/máximo
            \item Categorías específicas
            \item Proyecto asociado
        \end{itemize}
        \item Paginación para resultados extensos
    \end{itemize}
    
    \item \textbf{Visualización}:
    \begin{itemize}
        \item Tablas interactivas con ordenamiento por columnas
        \item Resúmenes gráficos usando datos de SupaBase
        \item Vista consolidada por periodos (diario, semanal, mensual)
        \item Sincronización en tiempo real con SupaBase subscriptions
    \end{itemize}
    
    \item \textbf{Validaciones}:
    \begin{itemize}
        \item Restricción de tipos de datos a nivel de base de datos
        \item Verificación de consistencia financiera
        \item Prevención de duplicados
    \end{itemize}
\end{itemize}

\subsection{Reportes y Finanzas}
\begin{itemize}
    \item Generación de reportes financieros
    \item Exportación a formatos estándar (Excel/PDF)
    \item Visualización de balances y gráficos
\end{itemize}
 % Sin extensión .tex
% 
% insertamos el como usar las funciones
\section{Uso de las Funcionalidades}

\subsection*{Gestión de Usuarios}
\begin{itemize}
    \item El usuario accede al formulario de registro e ingresa su correo y contraseña. Un correo de verificación es enviado automáticamente.
    \item Para iniciar sesión, el usuario proporciona sus credenciales previamente registradas. Si son válidas, se establece una sesión persistente.
    \item En caso de olvidar la contraseña, el usuario puede solicitar un enlace de recuperación, el cual expirará después de un tiempo definido.
    \item Desde su perfil, el usuario puede editar su información básica, como nombre o imagen, con cambios guardados automáticamente en Supabase.
    \item El último inicio de sesión del usuario se registra automáticamente mediante triggers en la base de datos.
\end{itemize}

\subsection*{Gestión de Ingresos y Egresos}
\begin{itemize}
    \item Desde el panel principal, el usuario puede registrar nuevas transacciones, completando los campos obligatorios como monto, categoría y fecha.
    \item Las transacciones están vinculadas a un proyecto específico mediante su \texttt{project\_id}.
    \item Las validaciones se realizan en tiempo real: si hay datos inválidos, el sistema los detecta antes de enviarlos a la base de datos.
    \item Para editar o eliminar una transacción, el usuario accede a su historial y selecciona la acción deseada. Las operaciones están protegidas por políticas RLS.
    \item En lugar de eliminar físicamente los datos, se aplica un \textit{soft delete} cambiando el valor de \texttt{deleted\_at}. Esto permite recuperar registros si es necesario.
    \item Cada modificación queda registrada en la tabla \texttt{transactions\_log}, lo cual facilita auditorías y revisiones posteriores.
    \item El sistema permite aplicar filtros avanzados para visualizar datos según fechas, categorías, montos o proyectos, utilizando funciones RPC.
    \item Los resultados se muestran en tablas interactivas y gráficos resumidos, con actualizaciones en tiempo real cuando hay nuevos cambios.
\end{itemize}

\subsection*{Reportes y Finanzas}
\begin{itemize}
    \item Desde el menú de reportes, el usuario puede generar balances por periodos determinados (diario, mensual, anual) mediante funciones RPC.
    \item Los resultados pueden exportarse en formatos como Excel o PDF para análisis externo o presentación \textbf{(pendiente)}.
    \item Los gráficos financieros permiten una visualización rápida del estado general de los ingresos y egresos por proyecto.
\end{itemize}

\subsection*{Gestión de Notificaciones}
\begin{itemize}
    \item El usuario recibe notificaciones relacionadas con sus proyectos y transacciones, las cuales se registran en la tabla \texttt{notifications}.
    \item Las notificaciones pueden ser marcadas como leídas mediante una función RPC.
    \item Las notificaciones expiradas se actualizan automáticamente mediante triggers en la base de datos.
\end{itemize}

\subsection*{Gestión de Adjuntos}
\begin{itemize}
    \item El usuario puede subir archivos relacionados con proyectos y transacciones, como facturas o documentos de soporte.
    \item Los adjuntos se validan para garantizar que cumplan con los formatos permitidos (PDF, JPG, PNG).
    \item Los archivos se almacenan en la tabla \texttt{attachments} y pueden ser consultados mediante funciones RPC.
\end{itemize} % Sin extensión .tex
% 

\section{Enlaces y Referencias}
\begin{itemize}
    \item Repositorio en GitHub: \url{https://github.com/Pattoxd45/~Plataforma-para-Control-de-Ingresos-y-Egresos-por-Proyecto}
    \item Enlace de despliegue: (pendiente de implementación)
\end{itemize}

\section{Conclusión}
Fintrax busca empoderar a los emprendedores mediante una herramienta simple pero poderosa para la gestión financiera temprana de proyectos. Este informe representa la primera fase del proyecto, donde se han establecido las bases técnicas y funcionales para el desarrollo posterior.

\end{document}

