\section{Autoevaluación del Cumplimiento}

\subsection*{Checklist de Tareas}
A continuación, se presenta una tabla con las tareas propuestas, su estado actual y observaciones:

\begin{table}[H]
\centering
\begin{tabular}{|p{5cm}|p{3cm}|p{6cm}|}
\hline
\textbf{Tarea} & \textbf{Estado} & \textbf{Observaciones} \\ \hline
Crear base de datos estructurada & Completado & Tablas principales creadas (\texttt{projects}, \texttt{transactions}, etc.). \\ \hline
Desarrollar frontend interactivo & En progreso & Componentes principales listos, falta optimización para móviles. \\ \hline
Integrar Supabase como backend & Completado & Autenticación, almacenamiento y funciones avanzadas configuradas. \\ \hline
Crear filtros avanzados para transacciones & Pendiente & Planificado para la próxima semana. \\ \hline
Exportación de reportes a Excel/PDF & Pendiente & Funcionalidad aún no implementada. \\ \hline
Realizar pruebas de integración & En progreso & Pruebas iniciales realizadas, falta cobertura completa. \\ \hline
Preparar entorno de despliegue & Pendiente & Configuración aún no iniciada. \\ \hline
\end{tabular}
\caption{Checklist de Tareas y Estado Actual}
\end{table}

\subsection**{Porcentaje de Avance}
El porcentaje de avance se calcula en función de las tareas completadas respecto al total:
\begin{itemize}
    \item Tareas completadas: 2
    \item Tareas en progreso: 2
    \item Tareas pendientes: 3
    \item \textbf{Porcentaje de avance:} \( \frac{3}{7} \times 100 = 42.9\% \)
\end{itemize}

\subsection*{Justificación de Tareas Pendientes}
Las tareas pendientes se deben principalmente a:
\begin{itemize}
    \item Falta de tiempo para implementar funcionalidades avanzadas como filtros y exportación.
    \item Coordinación limitada entre los miembros del equipo debido a la carga académica y otras responsabilidades.
\end{itemize}