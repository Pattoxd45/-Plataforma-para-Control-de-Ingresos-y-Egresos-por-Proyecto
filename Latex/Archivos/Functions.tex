\section{Funcionalidades}
\subsection*{Gestión de Usuarios}
\begin{itemize}
    \item \textbf{Autenticación} mediante Supabase Auth:
    \begin{itemize}
        \item Registro con verificación de email.
        \item Inicio de sesión con email/contraseña.
        \item Cierre de sesión con invalidación de tokens JWT.
        \item Sesiones persistentes seguras.
    \end{itemize}
    
    \item \textbf{Perfil de usuario}:
    \begin{itemize}
        \item Almacenamiento en tabla \texttt{auth.users} de Supabase.
        \item Edición segura de información básica.
        \item Registro de último inicio de sesión mediante triggers.
    \end{itemize}
\end{itemize}

\subsection*{Gestión de Ingresos y Egresos}
\begin{itemize}
    \item \textbf{Registro de transacciones}:
    \begin{itemize}
        \item Implementado mediante tablas \texttt{transactions} en Supabase.
        \item Campos obligatorios: monto, fecha, descripción, categoría, tipo (ingreso/egreso).
        \item Relación con proyectos mediante \texttt{project\_id}.
        \item Validación de presupuesto mediante triggers.
    \end{itemize}
    
    \item \textbf{Edición y eliminación}:
    \begin{itemize}
        \item \textbf{Protección de datos}:
        \begin{itemize}
            \item Implementación de Row Level Security (RLS) para restringir acceso.
            \item Políticas personalizadas para UPDATE/DELETE en Supabase.
        \end{itemize}
        
        \item \textbf{Soft Delete}:
        \begin{itemize}
            \item Campo \texttt{deleted\_at} (TIMESTAMPTZ) en tabla \texttt{transactions}.
            \item Valor por defecto NULL (registro activo).
            \item Índice parcial para optimizar queries: \texttt{WHERE deleted\_at IS NULL}.
            \item Columna adicional \texttt{deletion\_reason} (TEXT) para registrar motivo de eliminación.
        \end{itemize}
        
        \item \textbf{Histórico de cambios}:
        \begin{itemize}
            \item Tabla \texttt{transactions\_log} para auditoría.
            \item Triggers en Supabase que registran:
            \begin{itemize}
                \item Usuario que realizó el cambio.
                \item Fecha y hora exacta.
                \item Valores anteriores y nuevos (para operaciones UPDATE/DELETE).
            \end{itemize}
        \end{itemize}
    \end{itemize}
    
    \item \textbf{Filtrado avanzado}:
    \begin{itemize}
        \item Uso de funciones RPC para filtros dinámicos.
        \item Posibilidad de filtrar por:
        \begin{itemize}
            \item Rango de fechas.
            \item Monto mínimo/máximo.
            \item Categorías específicas.
            \item Proyecto asociado.
        \end{itemize}
        \item Paginación para resultados extensos.
    \end{itemize}
    
    \item \textbf{Visualización}:
    \begin{itemize}
        \item Tablas interactivas con ordenamiento por columnas.
        \item Vista consolidada por periodos (diario, semanal, mensual).
        \item Sincronización en tiempo real con Supabase subscriptions.
    \end{itemize}
\end{itemize}

\subsection*{Reportes y Finanzas}
\begin{itemize}
    \item Generación de reportes financieros mediante funciones RPC.
    \item Exportación a formatos estándar (Excel/PDF) \textbf{(pendiente)}.
\end{itemize}

\subsection*{Gestión de Notificaciones}
\begin{itemize}
    \item Registro de notificaciones en tabla \texttt{notifications}.
    \item Función RPC para marcar notificaciones como leídas.
    \item Trigger para marcar notificaciones expiradas automáticamente.
\end{itemize}

\subsection*{Gestión de Adjuntos}
\begin{itemize}
    \item Almacenamiento de archivos relacionados con proyectos y transacciones.
    \item Validación de formato de archivos mediante triggers.
    \item Función RPC para obtener adjuntos de un proyecto.
\end{itemize}