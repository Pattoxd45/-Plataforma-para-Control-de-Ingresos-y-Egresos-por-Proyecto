\section{Uso de las Funcionalidades}

\subsection{Gestión de Usuarios}
\begin{itemize}
    \item El usuario accede al formulario de registro e ingresa su correo y contraseña. Un correo de verificación es enviado automáticamente.
    \item Para iniciar sesión, el usuario proporciona sus credenciales previamente registradas. Si son válidas, se establece una sesión persistente.
    \item En caso de olvidar la contraseña, el usuario puede solicitar un enlace de recuperación, el cual expirará después de un tiempo definido.
    \item Desde su perfil, el usuario puede editar su información básica, como nombre o imagen, con cambios guardados automáticamente en SupaBase.
\end{itemize}

\subsection{Gestión de Ingresos y Egresos}
\begin{itemize}
    \item Desde el panel principal, el usuario puede registrar nuevas transacciones, completando los campos obligatorios como monto, categoría y fecha.
    \item Las transacciones pueden estar vinculadas a un proyecto específico mediante su \texttt{id\_proyecto}.
    \item Las validaciones se realizan en tiempo real: si hay datos inválidos, el sistema los detecta antes de enviarlos a la base de datos.
    \item Para editar o eliminar una transacción, el usuario accede a su historial y selecciona la acción deseada. Las operaciones están protegidas por políticas RLS.
    \item En lugar de eliminar físicamente los datos, se aplica un \textit{soft delete} cambiando el valor de \texttt{deleted\_at}. Esto permite recuperar registros si es necesario.
    \item Cada modificación queda registrada en la tabla \texttt{transacciones\_log}, lo cual facilita auditorías y revisiones posteriores.
    \item El sistema permite aplicar filtros avanzados para visualizar datos según fechas, categorías, montos o proyectos.
    \item Los resultados se muestran en tablas interactivas y gráficos resumidos, con actualizaciones en tiempo real cuando hay nuevos cambios.
\end{itemize}

\subsection{Reportes y Finanzas}
\begin{itemize}
    \item Desde el menú de reportes, el usuario puede generar balances por periodos determinados (diario, mensual, anual).
    \item Los resultados pueden exportarse en formatos como Excel o PDF para análisis externo o presentación.
    \item Los gráficos financieros permiten una visualización rápida del estado general de los ingresos y egresos por proyecto.
\end{itemize}
