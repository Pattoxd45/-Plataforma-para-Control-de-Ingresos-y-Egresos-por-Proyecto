\section{Código Fuente}

\subsection*{Fragmentos Relevantes}
A continuación, se presentan fragmentos de código funcionales que son esenciales para el funcionamiento de la plataforma, junto con una breve explicación de su propósito y funcionamiento.

\subsubsection*{Autenticación de Usuarios}
El siguiente código verifica si un usuario está autenticado y redirige al inicio de sesión si no hay una sesión activa:
\begin{lstlisting}[language=JavaScript]
import { useEffect } from 'react';
import { useNavigate } from 'react-router-dom';
import { supabase } from '../connections/endpoints';

const useAuthCheck = () => {
  const navigate = useNavigate();

  useEffect(() => {
    const checkSession = async () => {
      const { data: session } = await supabase.auth.getSession();
      if (!session?.session?.user) {
        navigate('/');
      }
    };

    checkSession();
  }, [navigate]);

  return { logout };
};
\end{lstlisting}
\textbf{Propósito:} Garantizar que solo usuarios autenticados puedan acceder a las funcionalidades de la plataforma.
\newpage
\subsubsection*{Gestión de Proyectos}
Este fragmento obtiene los proyectos asociados al usuario autenticado:
\begin{lstlisting}[language=JavaScript]
const fetchProjects = async (userId) => {
  try {
    const data = await endpoints.projects.getById(userId);
    const userProjects = data.filter((project) => project.user_id === userId);
    setProjects(userProjects);
  } catch (error) {
    console.error('Error fetching projects:', error);
    setErrorMessage('Error inesperado al obtener los proyectos.');
  } finally {
    setLoading(false);
  }
};
\end{lstlisting}
\textbf{Propósito:} Recuperar y mostrar los proyectos del usuario en el frontend.

\subsubsection*{Generación de Reportes Financieros}
El siguiente código utiliza una función RPC para generar reportes financieros:
\begin{lstlisting}[language=JavaScript]
const generateReport = async (projectId, reportType, startDate, endDate) => {
  const { data, error } = await supabase.rpc('generate_financial_report', {
    project_id: projectId,
    report_type: reportType,
    start_date: startDate,
    end_date: endDate,
  });
  if (error) throw error;
  return data;
};
\end{lstlisting}
\textbf{Propósito:} Generar reportes financieros personalizados basados en los datos de un proyecto.

\subsubsection*{Validación de Transacciones en la Base de Datos}
Este trigger en PostgreSQL valida que las transacciones no excedan el presupuesto del proyecto:
\begin{lstlisting}[language=SQL]
CREATE OR REPLACE FUNCTION public.validate_transaction_budget()
RETURNS TRIGGER AS $$
BEGIN
    IF (NEW.type = 'egreso') THEN
        IF (
            SELECT COALESCE(SUM(CASE WHEN type = 'ingreso' THEN amount ELSE -amount END), 0)
            FROM public.transactions
            WHERE project_id = NEW.project_id AND deleted_at IS NULL
        ) < NEW.amount THEN
            RAISE EXCEPTION 'El monto excede el presupuesto del proyecto.';
        END IF;
    END IF;
    RETURN NEW;
END;
$$ LANGUAGE plpgsql;
\end{lstlisting}
\textbf{Propósito:} Garantizar que las transacciones de tipo "egreso" no superen el presupuesto disponible del proyecto.

\subsection*{Conclusión}
Estos fragmentos de código representan las funcionalidades clave de la plataforma, desde la autenticación de usuarios hasta la gestión de proyectos y validaciones en la base de datos. Cada uno de ellos contribuye al correcto funcionamiento del sistema y asegura la integridad de los datos.